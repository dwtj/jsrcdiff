\chapter{Definitions}

We define a key and a value type for Java fields and methods. The key is
used to index a particular object, and the value is used to check for
equality. We define field and method keys and values here explicitly, as these
definitions directly influence program behavior and output.

\section{Fields}

\subsection{Key}

We define a field key to be:

\begin{itemize}
    \item the field type
    \item the field identifier (e.g. variable name)
\end{itemize}

\subsection{Value}

We deine a field value to be:

\begin{itemize}
    \item the field qualifiers (e.g. "public", "static", "final")
    \item the field initializers
\end{itemize}

\section{Methods}

\subsection{Key}

We define a method key as:

\begin{itemize}
    \item the method return type
    \item the method ID (i.e. method name)
    \item an ordered listed of the method argument types (e.g. "int, String, double")
\end{itemize}

\subsection{Value}

We define a method value as:

\begin{itemize}
    \item an ordered list of the method's argument IDs
    \item the method's qualifiers (e.g. "public")
    \item the text of the method body
\end{itemize}

Note in particular that we define the method value to include the \textit{entire}
method body; this implies that two methods with, for instance, one extra
space will be recognized as different. We do not do a semantic comparison of
method bodies. Further note that although we specify a more granular
definition of a method value, the assignment only calls for direct comparisons
of method bodies. Therefore, when testing for method equality, all value fields
are ignored except for the text of the method body.
